\begin{conclusions}
    \begin{enumerate}
        \item Con la investigación se evidenció que es factible implementar un modelo de simulación donde se tenga en cuenta la dinámica del comportamiento de las personas y los vectores en un entorno determinado. 
        \item Se confirmó que el uso de las redes complejas permite modelar de manera precisa el entorno descrito por localizaciones y relaciones entre agentes. Además se confirma, que tratar a las personas como agentes garantiza la posibilidad de ejecutar acciones de forma independiente.
        \item Se demostró que el uso de los $Mapas$ $Cognitivos$ $Difusos$ proporciona la base fundamental para que las decisiones de las personas como agentes estén debidamente orientadas.
        \item Los resultados obtenidos en la validación respaldan la utilidad y la efectividad del modelo propuesto como una herramienta confiable para el análisis de la propagación de enfermedades transmitidas por vectores.
    \end{enumerate}
\end{conclusions}
