\begin{resumen}
	En el presente trabajo, se realizó una investigación con el objetivo de implementar 
    un modelo basado en redes complejas para simular la dinámica de personas y vectores 
    en un entorno. Los objetivos específicos incluyeron el estudio y la comparación de 
    modelos existentes para la simulación de epidemias, el diseño de un modelo que 
    representara a las personas como agentes, la implementación de una aplicación que 
    permitiera simular diferentes escenarios y la validación del modelo desarrollado.

    En este estudio, se logró simular el comportamiento de las personas en la sociedad actual 
    al representar sus principales acciones en el modelo. Además, se modeló el entorno de 
    convivencia como una red compleja, donde las relaciones entre las personas se 
    representaron mediante aristas en un grafo. Estos logros cumplen con el objetivo general 
    de la investigación. También, se modeló a las personas como agentes para analizar su toma 
    de decisiones. Se desarrolló un Mapa Cognitivo Difuso que representaba las acciones como 
    conjuntos difusos, lo que permitió que los agentes decidieran entre diferentes acciones 
    en función de sus sentimientos y el grado de pertenencia a los conjuntos.

    Adicionalmente, se creó una aplicación de escritorio para facilitar la manipulación y 
    visualización del modelo desarrollado, permitiendo realizar simulaciones con los valores 
    de los parámetros que el usuario entienda. Los resultados fueron validados 
    utilizando estadísticas de trabajos previos relacionados con la simulación de epidemias, 
    lo que respalda la solidez de los resultados obtenidos y el cumplimiento de los objetivos 
    específicos propuestos.

    Esta investigación realiza una contribución significativa al campo de la simulación de 
    epidemias. Los resultados demuestran la utilidad de los modelos basados en 
    agentes para simular este tipo de eventos y pueden proporcionar información relevante para las 
    autoridades encargadas del control de epidemias, ya que, conociendo el comportamiento de una 
    epidemia en determinado momento, se pueden sugerir acciones a las personas y se facilita la 
    toma de decisiones para implementar medidas que reduzcan los riesgos asociados.

	\textbf{Palabras Claves:} Simulación, Redes Complejas, Modelos basados en agentes, Mapas Cognitivos Difusos,
	Conjuntos Difusos, Agentes, Vectores, Epidemias, Patógenos.
\end{resumen}

\begin{abstract}
	In the present study, research was conducted with the objective of implementing a complex 
	network-based model to simulate the dynamics of individuals and vectors in an environment. 
	The specific objectives included studying and comparing existing models for epidemic simulation, 
	designing a model that represented individuals as agents, implementing an application to simulate 
	different scenarios, and validating the developed model.

	In this thesis, the behavior of individuals in current society was successfully simulated by 
	representing their key actions in the model. Additionally, the social environment was modeled 
	as a complex network, where relationships between individuals were represented as edges in a 
	graph. These achievements fulfill the overall objective of the research. Furthermore, individuals 
	were modeled as agents to analyze their decision-making processes. A Fuzzy Cognitive Map was 
	developed, representing actions as fuzzy sets, which allowed agents to choose between different 
	actions based on their feelings and degree of membership to the sets.

	Moreover, a desktop application was created to facilitate the manipulation and visualization of 
	the developed model, enabling simulations with user-defined parameter values. The results were 
	validated using statistics from previous works related to epidemic simulation, supporting the 
	robustness of the obtained results and the fulfillment of the specific objectives.

	This research makes a significant contribution to the field of epidemic simulation. The results 
	demonstrate the utility of agent-based models in simulating such events and can provide relevant 
	information for authorities responsible for epidemic control. By understanding the behavior of an 
	epidemic at a given moment, actions can be suggested to individuals, and decision-making is 
	facilitated to implement measures that reduce associated risks.

	\textbf{Key Words:} Simulation, Complex Networks, Agent-Based Models, Fuzzy Cognitive Maps, Fuzzy Sets, 
	Agents, Vectors, Epidemics, Pathogens.
\end{abstract}