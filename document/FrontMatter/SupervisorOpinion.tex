\begin{opinion}
    El trabajo de diploma Simulación computacional para la dinámica de enfermedades transmitidas por vectores, presentado por el estudiante Ernesto Alfonso Hernández, para optar por el título de licenciado en Ciencia de la Computación, se corresponde con intereses del grupo de Modelación Biomatemática de la facultad de Matemática y Computación que trabajamos la modelación, solución y análisis de problemas aplicados a las biociencias.

Cuando hice propuestas para temas de diploma, entre mis aspiraciones estaba lograr resultados manejables para enfermedades trasmisibles utilizando redes complejas, ya que me pareció alentador lo aprendido a través de un trabajo anterior, de hace ya algunos años con una diplomante de la carrera de Matemática. El entusiasmo de Ernesto, ante estas alternativas, me ilusionó y desde el primer encuentro me pareció que tenía claro cómo trabajar.

Para lograr los objetivos que nos propusimos y que este trabajó superó, el diplomante realizó una investigación con el objetivo de implementar un modelo basado en redes complejas para simular la dinámica de transmisión del dengue entre personas y vectores en un determinado entorno, a través de la simulación computacional. Vale aclarar que hablamos de dengue, por las tareas en que el Grupo de Investigación está involucrado, pero que lo presentado es perfectamente válido para cualquier enfermedad transmitida por vectores.

Los que han participado en las defensas de otros trabajos que he dirigido, saben que por lo general repito que los estudiantes han trabajado con mucha independencia (especialmente con estudiantes de computación, pues mis conocimientos y posibilidades en temas de programación son muy limitados) y esta vez, tendré que repetirme pero además, con creces. 

Debo confesar que los pasos para lograr los objetivos específicos y que incluyeron el estudio y la comparación de modelos existentes para la simulación de epidemias, la selección y diseño de un modelo que representara a las personas como agentes, la implementación de una aplicación que simulara diferentes escenarios y permitiera la validación del modelo desarrollado, fueron trazados y transitados por Ernesto con una lógica incuestionable.

Y como si la estrategia diseñada no fuera ya en sí competente, Ernesto, gracias a estudios independientes y muy actuales, me propuso trabajar con mapas cognitivos difusos que representan las acciones como conjuntos difusos, lo que permitió que los agentes tuviera opciones de decisión para sus acciones, en función de su percepción del medio.

Además de este trabajo con importantes novedades teóricas y conceptuales, que por supuesto, requieren de conocimientos que no están contenidos en las asignaturas del currículo de la carrera, me toca agradecer la creación de una aplicación de escritorio para facilitar la manipulación y visualización del modelo desarrollado, con posibilidades de variantes que puede manejar el usuario. 

Como afirma el propio autor en la Introducción de su trabajo, esta investigación realiza una contribución significativa al campo de la simulación de epidemias, con resultados que demuestran la validez de los modelos basados en agentes y pudiera convertirse en una herramienta para los decisores de Salud en nuestro país.

Ha sido y lo sabemos, poco el tiempo con que han contado los diplomantes en este atípico último año de su carrera, para concluir sus trabajos de tesis, por ende, valoro doblemente los resultados obtenidos.

Considero que el trabajo presentado cumple con creces los requerimientos para ser defendido como tesis de licenciatura, con una evaluación excelente. Quería destacar de manera adicional, que me ha sorprendido muy favorablemente la maestría para escribir y redactar de Ernesto. 

Toda obra humana es perfectible y por supuesto que hay mucho que perfeccionar para que esta herramienta cumpla con mayores expectativas, pero estoy segura, porque lo ha demostrado sin lugar a dudas, que Ernesto tiene capacidad y condiciones para empeños mayores.

Por mi parte, lo invito a que mantengamos el mismo intercambio ameno y fructífero para dar continuidad al mismo y a trabajos futuros.

Le agradezco el trabajo conjunto, le deseo grandes éxitos en su vida profesional y lo felicito de corazón.

\begin{tabular}{r l}
    \includegraphics[width=0.2\linewidth]{Graphics/Opinion_Firma.png}\\
    \hrulefill \\
    Dra. Aymée Marrero Severo\\                    
    Tutora                                                        
\end{tabular}



\end{opinion}