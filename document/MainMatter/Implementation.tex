\chapter{Detalles de Implementación y Experimentos}\label{chapter:implementation}
La aplicación resultane de esta investigación está implementada en Visual Code, en Python 3.11.5
debido a la experiencia en el lenguaje y a la no especificación de requerimientos de programación. Cabe 
recalcar que el diseño del proyecto se puede realizar en casi cualquier lenguaje y plataforma.

\section{Detalles de Implementación}
El proyecto se encuentra dividido en varios módulos que se pueden representar en dos secciones.
La primera sería la sección de la lógica del proyecto, en esta se encuentran los módulos dedicados 
a la lógica de los agentes, la creación del entorno\footnote{Grafo de localizaciones y de relaciones}
y un último que es el encargado de realizar las simulaciones. La segunda sección es la encargada de 
la parte visual del proyecto, posee un único módulo denominado $frontend$\footnote{Termino utilizado en 
Ingeniería de Software para referirse a la parte visual de un aplicación} el cual es el designado para permitirle
al usuario interactuar con la aplicación; es decir, introducir las variables del modelo, simular, 
graficar los resultados obtenidos, entre otras acciones.

El conocimiento de la complejidad temporal del modelo generado permite entender la mejor manera de usarlo.
Esta complejidad temporal está dada por:
\begin{center}
    Sea $h$ la cantidad de horas a simular.\\
    Sea $p$ la cantidad de personas incluidas en la simulación.\\
    Sea $m_l$ la cantidad de mosquitos por localizaciones escogidos para picar en esa hora.\\
    Entonces la complejidad temporal del modelo es $O(h \times p \lbrack 3 \times 13 \times 16 + 14 + m_l \rbrack)$
\end{center}

Es fácil apreciar que si $h = p$ en nuestra simulación entonces la complejidad pasaría a ser cuadrática, ya 
que sería $p^2$. En el caso de que $m_l = p$, sería cuadrática también, pero $m_l$ es un valor 
que depende de una probabilidad por lo tanto el modelo no siempre procesaría el conjunto de mosquitos completos,
por lo que, para el peor caso\footnote{Peor caso: siempre se escogen todos los mosquitos del lugar para que piquen} demoraría
lo mismo, pero por lo general, sería más rápido.



\section{Experimentación}
Para la validación del modelo implementado se hizo necesario generar varios grafos con distintos parámetros.
En \autocite{Arazoza2010} se proponen unos valores fijos para ciertas variables de interés de este proyecto.\\

\textbf{Datos de estimaciones anteriores:}\autocite{Arazoza2010}
\begin{itemize}
    \item Tasa de mortalidad de humanos (0.000024).
    \item Tasa de recuperación de la enfermedad (0.143).
    \item Tasa de transmisión de humano a mosquito (entre 0.16346 y 0.16384).
\end{itemize}

La correcta evolución del modelo propuesto depende en gran medida de los valores de los parámetros que 
afectan al modelo y que, en su mayoría, se han considerado como constantes\footnote{Constantes para una 
instancia de simulación, pues pueden variar de una a otra}. La obtención de valores 
adecuados para estos parámetros es esencial para garantizar que el modelo refleje con precisión el 
fenómeno que se está estudiando. Para lograrlo, se requiere la contribución de expertos en áreas específicas 
de la ciencia que posean conocimientos y experiencia en la caracterización y medición de los parámetros.
Para grafos semejantes o isomorfos el modelo no necesariamente brinda resultados parecidos. Esto no solo 
se debe a la estocasticidad con la que se define el mismo, puede producirse también por valores de 
parámetros distintos; pero, grafos semejantes, con valores iguales de parámetros brindarían 
resultados muy parecidos. Con la idea de reflejar esto se realizaron simulaciones con valores distintos
para cada variable en cuestión.

Una idea seguida para la experimentación se basa en:
\begin{center}
    Sean $G_1=(V_1, E_1)$ y $G_2=(V_2, E_2)$ grafos de la simulación.\\
    $V_1 > V_2$ y $E_1 > E_2$.\\
    Sea $C_1$, $C_2$ el parámetro que define la conectividad de $G_1$, $G_2$ respectivamente.\\
    Si $C_1 = C_2$ se podría decir que estos grafos describen el mismo problema, es decir
    la curva de infección de $G_1$ sería similar a la de $G_2$.
\end{center} 

Por como se modeló el proyecto no solo influye el grado de conectividad para la infección o no de las personas,
también influyen las acciones que realizan y la estocasticidad debido a los parámetros de 
infección\footnote{Probabilidad de picar y Probabilidad de infectarse}, pero dejando a un lado estos dos 
últimos puntos se puede afirmar que se cumple la implicación anterior.

Debido a esto se entiende que es suficiente simular el proceso con una cantidad considerable de personas,
pero no es necesario hacerlo para el número exacto que se quiere simular. Por ejemplo,
si se desea simular para $p = 1000000$\footnote{$p$ cantidad de personas} en caso de que para $p = 1000$ 
se encontrara un grado de conectividad similar al del primer grafo, entonces la curva de infección
de este segundo, describiría la del primero. Sin embargo, si se cuenta 
con el tiempo necesario, podría ser útil hacerlo para el número de personas que se quiera. 

Se presentan los resultados a continuación:
