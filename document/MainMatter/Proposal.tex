\chapter{Modelación para la simulación de enfermedades transmitidas por vectores}\label{chapter:proposal}
\section{Vectores en la naturaleza}
Un vector es un organismo vivo, como un insecto o artrópodo, que puede transmitir un agente 
infeccioso, como un virus, una bacteria o un parásito, de un huésped infectado a un huésped 
susceptible. Los vectores pueden actuar como intermediarios en la transmisión de enfermedades, 
ya sea mecánicamente, a través de la contaminación de alimentos o superficies con el agente 
infeccioso, o biológicamente, cuando el agente infeccioso se replica y multiplica dentro del 
vector antes de ser transmitido a un nuevo huésped. Los vectores son una parte integral de la 
epidemiología de muchas enfermedades infecciosas y desempeñan un papel crucial en su mantenimiento 
y propagación.\autocite{Reisen2010}\\

Según la OMS, las enfermedades airborne (arthropod-borne) representan 17 por ciento del total de 
las enfermedades infecciosas en el mundo, con 1,000 millones de casos y un millón de defunciones 
anuales \autocite{OMS2020}. Los vectores biológicos más comunes son los insectos hematófagos que 
al alimentarse de la sangre de un portador infectado, ingieren microorganismos patógenos que 
posteriormente inoculan a otro individuo.\\

\textbf{Características generales de los vectores:}\autocite{OMS2020}
\begin{enumerate}
    \item Especies específicas: Los vectores suelen ser especies específicas de insectos, 
    artrópodos u otros organismos. Por ejemplo, los mosquitos, las garrapatas, las pulgas y los 
    flebótomos son ejemplos comunes de vectores. 
    \item Capacidad de transmitir enfermedades: Los vectores tienen la capacidad de transmitir 
    agentes patógenos, como virus, bacterias o parásitos, de un huésped infectado a un huésped 
    susceptible. Esto puede ocurrir a través de la picadura o el contacto con el vector.
    \item Dependencia de los huéspedes: Los vectores dependen de la sangre u otros recursos de 
    sus huéspedes para alimentarse y reproducirse. Por lo tanto, su presencia y actividad están 
    estrechamente relacionadas con la disponibilidad de los huéspedes adecuados.
\end{enumerate}
\subsection{Enfermedades Transmitidas por Vectores (ETV)}
Entre las ETV que han aumentado en las últimas décadas están el 
paludismo o malaria, la fiebre hemorrágica por dengue, la esquistosomiasis, la tripanosomiasis americana 
o enfermedad de Chagas, la tripanosomiasis africana o enfermedad del sueño, la leishmaniasis, la fiebre 
amarilla, la encefalitis japonesa, la fiebre por zika  y la fiebre por chikungunya. Otras ETV menos frecuentes 
son la borreliosis o enfermedad de  Lyme y la enfermedad por el  virus del oeste del Nilo.\autocite{Torres2020}\\

\begin{figure}[htb]
    \centering
    \includegraphics[width=1\textwidth]{Graphics/EnfVec.jpeg}
    \caption{Algunas enfermedades transmitidas por vectores.\autocite{Tercero2008}}
\end{figure}

La distribución de las ETV está vinculada a una serie de factores complejos 
de naturaleza demográfica, ecológica, medioambiental y social. Estos factores incluyen:
\begin{enumerate}
    \item El fenómeno del calentamiento global y el consiguiente cambio climático, que permite la adaptación de los vectores a nuevas altitudes y la propagación de patógenos en regiones previamente no afectadas.
    \item La sobrepoblación y el hacinamiento en áreas específicas, lo cual propicia un aumento en la presencia de vectores y hospederos susceptibles, como perros y gatos.
    \item La densidad de población de los artrópodos vectores y la diversidad de especies presentes en un área determinada.
    \item La falta de medidas adecuadas de higiene a nivel personal, en viviendas y en comunidades.
    \item El incremento en la frecuencia y distancia de los viajes internacionales.
    \item La existencia de marginación, pobreza y urbanización descontrolada.
\end{enumerate}

Actualmente, la enfermedad transmitida por vector con mayor crecimiento mundial es el dengue. 
Al igual que el virus del dengue, el del Zika, el chikungunya y la fiebre amarilla son transmitidos 
por los mosquitos Aedes aegypti y Aedes albopticus. Más de 3900 millones de personas en más de 129 
países corren el riesgo de contraer dengue, y se estima que cada año se registran 96 millones de casos 
sintomáticos y 40 000 muertes.\autocite{OMS2020}\\

\section{Dengue}
El dengue es una enfermedad viral transmitida por mosquitos que representa un desafío significativo 
para la salud pública a nivel mundial. Esta enfermedad se encuentra principalmente en regiones tropicales 
y subtropicales, pero su alcance se ha extendido durante las últimas décadas, llegando a afectar a más de 100 
países. El complejo del Dengue está formado por cuatro serotipos: dengue1, dengue2, dengue3 y dengue4. La 
infección humana por un serotipo produce inmunidad para toda la vida contra la reinfección por ese serotipo, 
pero el individuo queda susceptible a los otros tres. \autocite{Simmons2012} \\

Una vez una persona es picada por un mosquito infectado, se produce un período de incubación en esta, que puede 
durar de 5 a 7 días, luego de esto aparcen los primeros síntomas. El humano se encuentra enfermo aproximadamente
durante 7 o 15 días. En el momento en el que surgen los primeros síntomas comienza el período crítico de 
transmisión. Si en este intervalo de tiempo un mosquito pica a esa persona, entonces tiene una probabilidad 
alta de adquirir la enferemedad y luego de 8 a 12 días el virus se aloja en las glándulas salibales del mosquito
y es en esta etapa en que el mosquito es capaz de transmitir la enfermedad. \autocite{OMS2023} \\

Una de las características más preocupantes del Dengue es su capacidad para causar una amplia gama de síntomas, 
desde una fiebre leve hasta formas más preocupantes que pueden poner en peligro la vida. La forma grave de la 
enfermedad, conocida como fiebre hemorrágica del dengue, puede producir hemorragias internas, 
disfunción orgánica y shock. Esta forma afecta principalmente a niños pequeños, 
adultos mayores y personas con sistemas inmunológicos debilitados.\\

\textbf{Síntomas del Dengue:}\autocite{OMS2023}
\begin{enumerate}
    \item Fibre alta ($40^{\circ}$C).
    \item Dolores de cabeza.
    \item Dolor detrás de los ojos.
    \item Náuseas
    \item Vómitos
    \item Rash
\end{enumerate}

El mosquito aede aegypti es el vector del Dengue. El ciclo de vida de un mosquito es aproximadamente entre 4 y 
8 semanas y ocurre en varias etapas, huevo, larva, pupa y adulto. Los que transmiten la enfermedad son los mosquitos
hembras adultos. Estas necesitan de sangre para desarrollar los huevos y pueden volar hasta 3 kilómetros con 
el objetivo de encontrar un lugar para ponerlos, aunque no se espera que vuelen más de 100 metros del citio 
donde viven.\\

\section{Modelación del entorno}
Lo primero que se debe crear y modelar para simular como ocurre la propagación de enfermedades es el 
entorno en que ocurrirá la misma. La idea de este trabajo es lograr representar la realidad de la sociedad
y para esto se entiende que existen parámetros a desarrollar: las relaciones personales, los lugares y el comportamiento
de las personas.\\

\subsection{Loaclizaciones}
En la actualidad la mayoría de las personas tienen un programa de vida definido, es decir, una persona $x$ tiene una 
vivienda, un centro de trabajo y otros lugares a los que asiste por determinadas circunstancias; por lo que
para representar la sociedad, es necesario según se entiende, modelar estos lugares.\\

Para esto, los lugares son considerados un objeto en nuestro proyecto. Se entiende por lugar una localización en
la cual pueden asistir personas y vectores y además este puede brindar cierto recurso. Por ejemplo: ¿cómo se
representaría un mercado en nuestra simulación?, un mercado posee una capacidad para albergar personas y vectores 
y además brinda la posibilidad de obtener comida, aseo, entre otros.\\

\textbf{Localizaciones relevantes a representar:}
\begin{enumerate}
    \item Casas: representa un hogar familiar.
    \item Hospitales: indica un centro de atención médica.
    \item Centros de trabajo: hace referencia a todo lugar laboral, pero no significa que si una persona está en este, se encuentra trabajando, por ejemplo: en una escuela (centro de trabajo) hay personas que no están trabajando como los niños.
    \item Mercados.
\end{enumerate}

Este tipo de localizaciones nos permite abarcar toda construcción en la que pueden relacionarce las personas, pués
el tipo de lugar $"Centro$ $de$ $Trabajo"$ sirve de comodín en nuestra simulación.\\

\subsection{Personas}
Una de las herramientas conocidas para describir relaciones entre agentes son los grafos \autocite{Newman2003}. Un grafo
es un par ordenado $G = (V,E)$ donde $V$ es un conjunto no vacío de nodos y $E$ es un conjunto de pares 
no ordenados de aristas.\\

\begin{center}
    $G=(V,E)$ tal que:\\
    \begin{enumerate}
        \item $V$ = $\lbrace$ $v_{1}$, $v_{2}$, $v_{3}$, ..., $v_{n}$ $\rbrace$ es un conjunto finito de vértices.
        \item $E$ = $\lbrace$($v_{i}$,$v_{j}$)|$v_{i}$, $v_{j}$ $\in$ $V$ $\rbrace$  es un conjunto de pares no ordenados de vértices que representan las aristas del grafo.
    \end{enumerate}
\end{center}

En nuestro proyecto se construye un grafo que representa la relación ser familia y la relación conocidos, brindando
la posibilidad de escoger la probabilidad con la que se genere una arista entre dos nodos y la cantidad de nodos a crear. 
En este, un nodo $v_{i}$ representa a la persona $i$ de la simulación y una arista ($v_{i}$, $v_{j}$) simboliza,
o bien la relación $i$ $\rightarrow$ $j$ son familiares, o son conocidos, es decir no existe un tipo de arista para cada
tipo de relación. ¿Cómo se identifica entonces si la arista ($v_{i}$, $v_{j}$) representa la relación 
ser familia o la relación conocidos?\\

Para esto se realiza un proceso estocástico que ocurre una sola vez (al inicio de la simulación), el cual consiste 
en escoger de manera aleatoria el conjunto $C$ que se define a continuación.
\begin{center}
    Sea $G = (V,E)$ un grafo de nuestra simulación.\\
    Sea $H \subset V$ tal que $v_{i} \in H \Leftrightarrow \forall v_{j} \in H, (v_{i},v_{j}) \in E$\\
    Sea $C = \lbrace H_{1}, H_{2}, ..., H_{l} \rbrace$,$\forall i$ $H_{i} \subset V$ tal que si $v_{j} \in H_{i} \Rightarrow v_{j} \notin H_{k}$ $\forall k \neq i$\\
\end{center}

Teniendo esto en cuenta entonces las relaciones están definidas de la siguiente forma:
\begin{center}
    $\forall i,j$ $(v_{i}, v_{j}) \in E$ representa la relación ser familia  $\Leftrightarrow \exists k$ tal que $v_{i}, v_{j} \in H_{k}$, $H_{k} \in C$.\\
    Si $v_{i} \in H_{k}, v_{j} \notin H_{k}$, $H_{k} \in C$ y $\exists (v_{i}, v_{j}) \Rightarrow$ la arista $(v_{i}, v_{j})$  representa
    la relación ser conocidos.
\end{center}


\begin{figure}[htb]
    \centering
    \includegraphics[width=0.8\textwidth]{Graphics/Grafo_Pers.png}
    \caption{Ejemplo de grafo de relaciones personales generados por el modelo. Personas: 300, probabilidad de arista: 0.05}
\end{figure}

\begin{figure}
    \centering
    \includegraphics[width=0.3\textwidth]{Graphics/Grafo_Familias.jpeg}
    \caption{Ejemplo de grafo de relaciones personales para ilustrar.}
\end{figure}

En la figura 2.3 se observa un posible grafo generado. En este grafo los posibles conjuntos $H$ serían: 
$H_{1} = (1,2)$; $H_{2}=(1,3)$;$H_{3}=(2,3)$; $H_{4} = (1,2,3)$; $H_{5} = (1,5)$; $H_{6}=(1,4)$; $H_{7} = (4,6)$; $H_{8} = (5,6)$ y por último tantos conjuntos $H$ como nodos haya, es decir
todos los posibles cliques del grafo. De estos conjuntos se seleccionarían aleatoriamente algunos para formar el conjunto $C$;
por ejemplo $C = \lbrace H_{4}, H_{7}, H_{15} \rbrace$; y los nodos que estan en los $H_{i} \in C$ serían considerados
familias entre ellos.\\

Las personas, en el curso de la realización de sus actividades diarias (como el trabajo, el estudio o las compras),
se desplazan entre varios lugares, exponiéndose a agentes infecciosos dentro de estos lugares y transportando las
enfermedades. Para lograr representar y modelar estos procesos se genera una red de contactos sociales que puede
ser vista como un grafo bipartito, en el cual el conjunto $A$ está compuesto por todas las personas de la simulación
y el conjunto $B$ por todas las localizaciones. Las aristas en este grafo son dirigidas y representan 
el lugar en donde se encuentra la persona.\\
\begin{center}
    Sea $G = (V,E)$ grafo dirigido. 
    $\forall i,j$ si $(v_i,v_j) \in E \Rightarrow$ la persona $v_i$ se encuentra en el lugar $v_j$ 
\end{center}


\begin{figure}[htb]
    \centering
    \includegraphics[width=0.3\textwidth]{Graphics/Grafo_Loc_Pers.jpeg}
    \caption{Ejemplo de grafo que representa a la red de contactos sociales.}
\end{figure}

El grafo definido anteriormente es un grafo dinámico, es decir este varía en dependencia del lugar en donde se encuentre
una persona, pues al decidir un agente ir a otro lugar, se elimina la arista que antes este tenía en el grafo y se 
añade la nueva arista, la cual representa el lugar en donde se encuentra en el momento. (Los vértices que representan
personas tienen $outdegree$ igual a 1)\\

\subsection{Vectores}
Los vectores son seres vivos al igual que las personas, pero a diferencia de estas estos poseen menos movilidad que las
personas. Para la representación de los vectores en el modelo no se tiene en cuenta la relación que 
poseen entre ellos pues se entiende que es suficiente con las relaciones de las personas para lograr simular 
la propagación de una enfermedad.\\ 

Los mosquitos como bien se mencionó anteriormente se establecen en un lugar y poco o nada se mueven de 
sus alrededores, por tanto, se decidió que los vectores en nuestra simulación no tuviesen la capacidad de moverse por
las localizaciones como las personas ya que esto nos acerca más a lo que ocurre en la realidad.\\

Los vectores se modelan con la posibilidad de morder y de infectarse. Estos según mecanismos estocástico
deciden si picar o no y teniendo en cuenta el nivel de infección de la persona y la susceptibilidad del mismo este
se infecta o no. También poseen un parámetro que representa la probabilidad de infectarse debido a una mordida, mientras
más alto más probable que este se infecte si pica o muerde a una persona enferma.\\

\begin{figure}[htb]
    \centering
    \includegraphics[width=0.9\textwidth]{Graphics/Pers_Loc_Vec.png}
    \caption{Esquema que representa el entorno modelado.}
\end{figure}
En la figura 2.4 se observa como se encuentra modelado el entorno de la simulación. Existen unos agentes que son las
personas y los vectores, los cuales poseen ciertas características y estos interactúan con las localizaciones para
cumplir sus propósitos; el de las personas trabajar y socializar y el de los vectores alimentarse, logrando así
acercarnos a un modelo que representa de forma precisa como se propaga un efermedad transmitida por vectores.

\section{Interacción de los agentes con el entorno}
Una vez diseñado el entorno y los agentes que conviven en el mismo, se abre paso al siguiente aspecto:
como interactúan entre sí. La interacción entre los agentes y su medio es esencial para comprender 
cómo se desarrollará y evolucionará el sistema en cuestión.\\
Existen muchas formas de desarrollar las interacciones, dependiendo del contexto y los objetivos establecidos
para los agentes. Pueden ser directas, donde los agentes interactúan entre sí de manera mutuamente perceptible,
indirectas, donde los agentes influyen en el entorno, y a su vez, el entorno afecta a la forma en que estos se 
comportan y un híbrido en el que los agentes según sus decisiones afectan al medio y a otros agentes, y también, 
el entorno los afecta.\\
Las interacciones pueden estar basadas en reglas predefinidas, donde se establecen normas, objetivos y una serie
de comportamientos específicos, o pueden ser adaptativas, donde los agentes aprenden a medida que intercambian con
el entorno, obteniendo retroalimentación del mismo, para modificar su comportamiento.\\
Una herramienta computacional que brinda la posibilidad de crear agentes con cierta inteligencia para manejar
sus decisiones son los Mapas Cognitivos Difusos (FCM). Para el diseño de un $FCM$ es necesario definir los 
conceptos que este agrupará, así como las categorías de conceptos.\\
¿Qué son los conjuntos difusos? Zadeh en \autocite{Zadeh1965} da respuesta a esta interrogante de la siguiente forma.
\begin{center}
    Sea $X$ un espacio de puntos en los que los elementos de $X$ son $x$. $X = \lbrace x \rbrace$
\end{center}
Un conjunto difuso $A$ en $X$ es caracterizado por una función de membresía $f_A(x)$ que asocia a cada punto en 
$X$ un valor real en el intervalo $[0, 1]$ con el valor de $f_A(x)$ en $x$ representando el 
grado de membresía de $x$ en $A$, tal que mientras más cerca el valor de $f_A(x)$ a la unidad más alto es 
el grado de membresía de $x$ en $A$. Pongamos un ejemplo representado en \autocite{Zadeh1965}.\\
Sea $X$ el conjunto de los números reales $R$ y sea $A$ un conjunto difuso de los números que son mucho mayores que 1.
La función $f_A(X)$ podría tener ciertos valores representativos como: $f_A(0) = 0$, $f_A(1) = 0$,
$f_A(5) = 0.01$, $f_A(10) = 0.2$, $f_A(100) = 0.95$, $f_A(500) = 1$.\\

Es importante notar que cuando el conjunto $X$ es un conjunto contable la función de membresía es parecida a una 
función de probabilidad (o parecido a la función de densidad cuando $X$ es continuo), evidentemente existen
diferencias entre estos conceptos las cuales son especificadas por Zadeh en \autocite{Zadeh1965}.\\

% \textbf{Algunas definiciones de conjuntos difusos:}\autocite{Zadeh1965}\\
% \begin{itemize}
%     \item Se dice que dos conjuntos difusos $A$ y $B$ son iguales, $\Leftrightarrow$ $f_A(X) = f_B(x)$ $\forall x$ en $X$.
%     \item Un conjunto difuso $A$ es vacío $\Leftrightarrow$ $f_A(x) = 0$ $\forall x$ en $X$.
%     \item El complemento de un conjunto difuso $A$ es $A'$ y es definido de la siguiente forma: $f_{A'} = 1 - f_A$.
%     \item $A \subset B \Leftrightarrow f_A \leqq f_B$
% \end{itemize}

En el Capítulo $"Estado$ $del$ $Arte"$ se realiza un acercamiento a la posibilidad de crear un $FCM$ con tres clases de 
conceptos; $Perspectiva$, $Sentimientos$ y $Acciones$. Cada clase tiene una serie de conceptos que posibilitan la 
interacción entre clases.\\
Al separar los conceptos se comienza a ver el grafo del $FCM$ como un grafo bipartito, en el cual el conjunto $A$
se encuentra formado por los conceptos de las clases $Perspectiva$ y $Acciones$ y el conjunto $B$ por la clase
$Sentimientos$. Entonces surge una interrogante, ¿cuál sería el flujo a seguir de este $FCM$ para que los agentes
decidan una acción u otra?.\\
La idea es la siguiente, un agente percibe un estado del entorno, este estado, provoca un sentimiento en el agente
y, a su vez, este sentimiento provoca una acción.
\begin{center}
    $Perspectiva \rightarrow Sentimientos \rightarrow Acciones$\\
\end{center}

A medida que va cambiando el entorno, va cambiando la perspectiva del agente esta afecta a los sentimientos del mismo 
y con esto cambia la probabilidad de efectuar una acción u otra. Por tanto, el tipo de relación utilizada en este $FCM$
es híbrida, cada pequeño cambio, ya sea, en el entorno, o en el mapa cognitivo difuso de un agente, afecta al $FCM$ de los
agentes restantes.\\

\subsection{Conceptos (Personas)}
\subsubsection{Perspectiva}
Para la sección de los conceptos de percepción se observa que una idea útil es que, para cada concepto, crear su contrapuesto
como concepto, por ejemplo, si se definiese el concepto $"cercanía de hospital"$, entonces sería útil definir el concepto
$"lejanía de hospital"$ pués las posibles acciones a ejecutar se beneficiarían por uno y se perjudicarían por el otro,
añadiendo facilidad a la decisión del agente.\\
\\
\textbf{Conceptos definidos para la categoría Perspectiva:}
\begin{itemize}
    \item Personas enfermas alta. (1.4)
    \item Personas enfermas baja. (1.4)
    \item Comida alta. (1.5)
    \item Comida baja. (1.5)
    \item Energía alta. (3)
    \item Energía baja. (3)
    \item Dinero alto. (1.5)
    \item Dinero bajo. (1.5)
    \item Enfermedad alta. (3.2)
    \item Enfermedad baja. (3.2)
\end{itemize}

Cada concepto en esta categoría, tiene un parámetro que representa cuan grande es el intervalo que se considera para darle un valor en la 
$fuziificación$ entre $(0,1)$ al concepto (el número que se encuentra entre paréntesis al lado del concepto), en caso 
de que sea menor o mayor a los límites del intervalo, su valor final es $0$ o $1$ respectivamente, pero, ¿por qué
se realiza este procedimiento?\\

La idea de implentar de esta forma esta categoría es para $fuzzificar$ teniendo en cuenta lo que se percibe 
del entorno. El valor de este parámetro por concepto nunca cambia, entonces surge la interrogante siguiente: 
¿cómo modelar que la perspectiva del entorno varía? Este valor pasa por un proceso en el cual se toman variables que si
cambian en el entorno y se utiliza el mismo para $fuzzificar$ el valor del concepto, el cual es distinto al parámetro 
representado en la lista anterior.\\

El proceso de $fuzzificación$ es distinto para cada concepto, pero por regla general se sigue la siguiente idea:\\
\begin{enumerate}
    \item Se toma un valor del entorno del entorno del agente que tenga relación con el concepto a $fuzzificar$, llamémosle $variable$ $de$ $fuzzificación$.
    \item Se obtiene un intervalo de valores utilizando el parámetro del concepto, llamémosle $intervalo$ $de$ $fuzzificación$.
    \item Se compara la $variable$ $de$ $fuzzificación$ con los extremos del $intervalo$ $de$ $fuzzificación$, en el caso de encontrarse incluida en este se decide si es más importante que este cerca del máximo o del mínimo del $intervalo$ $de$ $fuzzificación$ y se $fuzzifica$ de acuerdo al intervalo, teniendo en cuenta cual extremo es considerado $1$ y cual $0$ en el proceso de $fuzzificación$.
\end{enumerate}

\begin{center}
    Sea $v$ la $variable$ $de$ $fuzzificación$.\\
    Sea $(i_0, i_1)$ el $intervalo$ $de$ $fuzzificación$.\\
    Sea $r$ el resultado del proceso de $fuzzificación$.\\
    Sea $inv$ una variable booleana, tal que: $if$ $inv = True, v \geq i_1 \Rightarrow r = 1$ , $if$ $inv = False, v \leq i_0 \Rightarrow r = 1$\\
\end{center}
    Cuando $v$ se encuentra dentro del intervalo ocurre lo siguiente:
\begin{center}
    $if$ $inv = True$ $\Rightarrow$ $r = \frac{v - i_0}{i_1 - i_0}$\\
    $if$ $inv = False$ $\Rightarrow$ $r = \frac{i_1 - v}{i_1 - i_0}$
\end{center}

La combinación de valores en estos conceptos, o el valor de un concepto por sí mismo, afecta al valor que toma algún
concepto en la categoría $Sentimientos$.\\

\subsection{Sentimientos}
En esta categoría se tienen en cuenta los sentimientos básicos de un ser humano y los que mejor se adaptaban para
representar la movilidad de los mismos.\\
\\
\textbf{Conceptos definidos para la categoría Sentimientos:}
\begin{itemize}
    \item Miedo.
    \item Hambre.
    \item Necesidad.
    \item Enfermedad.
    \item Indiferencia.
    \item Cansancio.
\end{itemize}

\subsection{Acciones}
Se definen las acciones fundamentales que modelan el comportamiento de una persona en un ambiente epidémico.\\
\\
\textbf{Conceptos definidos para la categoría Acciones:}
\begin{itemize}
    \item Ir a trabajar.
    \item Ir al mercado.
    \item Ir al hospital.
    \item Caminar.
    \item Estudiar.
    \item Descansar.
    \item Prevenir.
\end{itemize}

Definamos ahora los conceptos que tienen relación entre sí, es decir que conceptos influyen en el valor de otros.
Para esto será necesario ilustrar en una imagen el grafo resultante.

% "people_sick_high" : (0, 1.4),
%             "people_sick_low" : (1, 1.4),
%             "food_high" : (2, 1.5),
%             "food_low" : (3, 1.5),
%             "energy_high" : (4, 3),
%             "energy_low" : (5, 3),
%             "money_high" : (6, 1.5),
%             "money_low" : (7, 1.5),
%             "sickness_high" : (8, 3.2),
%             "sickness_low" : (9, 3.2)

% "fear" : 10,
%             # "loneliness" : 11,
%             "hunger" : 12,
%             "necessity" : 13,
%             "disease" : 14,
%             "indifference" : 15,
%             "tiredness" : 16


% "go_to_work" : 17,
%             "go_to_market" : 18,
%             "go_to_hospital" : 19,
%             "go_around" : 20,
%             "study" : 21,
%             "rest" : 22,
%             "prevent": 23