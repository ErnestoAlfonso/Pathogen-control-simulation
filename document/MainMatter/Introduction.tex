\chapter*{Introducción}\label{chapter:introduction}
\addcontentsline{toc}{chapter}{Introducción}
Las enfermedades infecciosas han sido una preocupación constante en todo el mundo debido 
a su impacto devastador en la salud humana y la sociedad en general. Estas enfermedades 
son causadas por agentes patógenos, como bacterias, virus, hongos, parásitos y vectores, que pueden 
transmitirse de una persona a otra. \\
A lo largo de la historia, las enfermedades infecciosas 
han desencadenado pandemias y epidemias, cobrando innumerables vidas y afectando la estabilidad 
de comunidades y naciones. Estas enfermedades pueden tener consecuencias a corto plazo, como 
enfermedades graves e incluso la muerte, así como impactos a largo plazo, como discapacidades y 
secuelas. Además del sufrimiento humano, las enfermedades infecciosas también tienen un impacto 
socioeconómico significativo, afectando la productividad, el desarrollo y los sistemas de atención 
médica de los países. A pesar de los avances en medicina y prevención, las enfermedades infecciosas 
continúan representando desafíos persistentes en todo el mundo. \\
El dengue es una infección transmitida por mosquitos que se presenta en todas
las regiones tropicales y subtropicales del planeta. En años recientes, la transmisión
ha aumentado de manera predominante en zonas urbanas y semiurbanas y se ha convertido 
en un importante problema de salud pública. La dinámica de la transmisión
de enfermedades infecciosas por vectores es compleja y depende de múltiples factores, 
incluyendo la biología del vector, la ecología del huésped y el entorno físico.\autocite{Hasan2016}\\ 
En cuanto a las estrategias de control, existen diversas opciones que se pueden
evaluar mediante la simulación computacional. Una de las estrategias más comunes es
el uso de insecticidas, que se aplican en áreas donde los vectores se reproducen y se
alimentan. Otra estrategia de control es la implementación de programas de prevención y
educación, que buscan reducir la exposición de las personas al vector y la enfermedad.
Por ejemplo, se pueden distribuir repelentes de insectos y mosquiteros para prevenir
las picaduras de mosquitos, y se pueden implementar campañas de educación para
promover prácticas seguras de eliminación de criaderos de mosquitos. \autocite{Gubler1995}\\
La simulación computacional es una herramienta esencial para estudiar la dinámica y 
el control de patógenos transmitidos por vectores. Al utilizar diferentes modelos
y técnicas de simulación, se pueden explorar diversos escenarios y estrategias de
control para prevenir y mitigar la propagación de enfermedades infecciosas. Estas 
simulaciones pueden ser una herramienta muy valiosa para los responsables de la toma
de decisiones en salud pública, ya que pueden ayudar a evaluar la efectividad de 
diferentes medidas de control y a prever el impacto de una enfermedad infecciosa \autocite{Epstein2008}, como
es el caso del dengue en nuestra población. \\

\section{Motivación}
Enfrentarse a una epidemia presenta una serie de desafíos significativos que requieren 
una respuesta rápida y coordinada. Estos desafíos pueden variar según la naturaleza del 
patógeno, la magnitud de la epidemia y el contexto socioeconómico en el que se produce, 
pero siempre el principal desafío es la pérdida de vidas humanas.\\
Otro de los principales problemas que genera una epidemia es el impacto económico y social. 
Las epidemias pueden tener un impacto significativo en la economía y el tejido social 
de las comunidades afectadas. Tanto las medidas de control, como los gastos médicos podrían 
traer consigo el inicio de una crisis económica.\\
Las enfermedades pueden propagarse rápidamente y colapsar los sistemas de atención 
médica. La capacidad de transmisión del patógeno puede dificultar la contención y el control 
eficaz de la enfermedad, especialmente si no se toman medidas preventivas adecuadas.\\
Enfrentar una epidemia requiere una sólida coordinación y colaboración entre diferentes 
entidades, como agencias de salud pública, gobiernos, organizaciones no gubernamentales y 
comunidades. La falta de coordinación puede dificultar la implementación de medidas de 
control y generar confusión entre la población.\\
Teniendo en cuenta estos desafíos, sería de gran importancia tener una herramienta 
computacional que simule estos escenarios para así ayudar a la toma de decisiones, atendiendo 
al conocimiento que esta podría brindar.\\

\section{Antecedentes}
La experiencia del Grupo de Biomatemática de la facultad de Matemática y Computación en el trabajo 
con modelos epidemiológicos poblacionales y la estimación y ajuste de sus parámetros, avala la 
importancia y necesidad de contar con herramientas computacionales que simulen y resuelvan los problemas 
inherentes a la modelación, solución, estimación y predicción.\\
En relación a esta temática existen trabajos como Redes Complejas en Epidemiología. Aplicaciones a 
modelos de VIH y Dengue su autora es Glenda Beatriz Rodríguez García, para optar por el título 
de licenciatura en Matemática, y de tutora: Dra. Aymée Marrero Severo (2016). En este se presentan
conceptos sobre la teoría de redes complejas y se analiza su aplicación al caso de epidemias de Dengue
y VIH. Además, utiliza un sistema de ecuaciones diferenciales para mostrar una comparativa con el 
resultado obtenido por el modelo de redes complejas.\\ 

\section{Problema de Investigación}
La simulación computacional es una herramienta poderosa para modelar y comprender
la dinámica de la transmisión de patógenos por vectores. Al utilizar la simulación,
se pueden explorar diferentes escenarios y estrategias de control para prevenir y mitigar 
la propagación de enfermedades.\autocite{Ferguson2006}\\
Existen diferentes tipos de modelos que se pueden utilizar para estudiar 
la dinámica de la transmisión de enfermedades infecciosas por vectores. Uno
de los modelos más comunes es el SIR 
\autocite{Kermack1927}, que divide la población en tres grupos:
susceptibles, infectados y recuperados. El modelo SIR se utiliza para entender cómo
se propaga una enfermedad infecciosa a través de una población y cómo la enfermedad 
puede ser controlada. Otros, incluyen modelos basados en
agentes, que simulan el comportamiento individual de los vectores y los huéspedes, y
modelos de redes, que modelan las interacciones entre los vectores, los huéspedes y
el entorno. Cada uno de estos métodos ofrece distintos recursos para enfrentar el proceso de simulación.
\autocite{Ferguson2006} \autocite{Balcan2009}\\
Teniendo en cuenta los diferentes métodos que existen, la problemática de la presente
investigación sería la modelación de una simulación para la dinámica de patógenos 
transmitidos por vectores, utilizando recursos que simulen adecuadamente la toma de 
decisiones de las personas.\\

\section{Pregunta Científica}
Por lo analizado durante el estudio del Estado del Arte se plantea la pregunta científica: ¿es posible implementar una
herramienta de simulación, la cual modele la propagación de una enfermedad transmitida por vectores 
teniendo en cuenta el comportamiento dinámico
de las personas?


\section{Objetivos}
\subsection{Objetivo general}
Implementar un modelo que, usando redes complejas, simule la dinámica de las personas y los vectores en el medio, 
desarrollando una herramienta de trabajo para el mismo.

\subsection{Objetivos específicos}
\begin{itemize}
    \item Estudiar y comparar los modelos que existen para la simulaciones de epidemias que se adapten a los requerimientos establecidos.
    \item Diseñar un modelo para la simulación que represente a las personas como agentes.
    \item Implementar una aplicación que permita simular diferentes escenarios del modelo desarrollado.
    \item Validar el modelo implementado con modelos clásicos.
\end{itemize}


\section{Estructura}
El documento se encuentra estructurado en tres capítulos. En el Capítulo 1 
se realiza un estudio sobre el marco teórico conceptual del problema en cuestión, haciendo enfásis en 
las Redes Sociales, la Modelación Basada en Agentes y los Mapas Cognitivos Difusos. El Capítulo 2 aborda sobre
la modelación de los agentes y el entorno, argumentando el mapa cognitivo perteneciente a los agentes. En el 
Capítulo 3 se especifican los aspectos técnicos de la implementación del modelo y se lleva a cabo un 
análisis del valor de la solución desarrollada mediante experimentos. Finalmente se presentan las Conclusiones,
que resume los hallazgos claves y los puntos principales, dando respuesta a los objetivos según los resultados obtenidos.


%\section{Propuesta de solución}
%Para la solución de la problemática se propone realizar un modelo basado en agentes, en el que las personas son
%tratadas como los agentes del mismo. Estas se desempeñan en una red social en la cual estan representadas
%las relaciones familia y amistad.\\
%Además se propone la implementación de otra red social para representar el lugar exacto en el que se encuentre una 
%persona. Esta es un grafo bipartito en el que los nodos de un conjunto son las personas y del otro son las localizaciones
%existentes en el medio. Esta es una red social dinámica en la cual todos los nodos tienen grado.\\
%Las personas en este ambiente realizan acciones a través de un Mapa Cognitivo Difuso, decidiendo, según su estado de 
%ánimo, como y cual acción ejecutar.\\
%Por último la apliación de escritorio brindará todo lo necesario para simular y analizar la simulación una vez terminada.



